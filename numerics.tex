\documentclass[twocolumn,aps,prl,superscriptaddress,english]{revtex4-2}

\usepackage{graphicx}% Include figure files
\usepackage{dcolumn}% Align table columns on decimal point
\usepackage{bm}% bold math
\usepackage{hyperref}% add hypertext capabilities
\usepackage{lipsum}
\usepackage{tikz,tikz-cd}
\usepackage{braket}
\usepackage{siunitx}
\usepackage{stmaryrd}
\usepackage{booktabs}

\usetikzlibrary{quantikz2}

\hypersetup{
colorlinks=true,
linkcolor=blue,
filecolor=blue,
citecolor=blue,  
urlcolor=blue,
}
\newcommand{\yw}[1]{\textcolor{purple}{YW: #1}}
\renewcommand{\tilde}{\widetilde}
\renewcommand{\leq}{\leqslant}
\renewcommand{\geq}{\geqslant}


\newcommand{\YG}[1]{{\color{red}[YG: #1]}}
\newcommand{\wyf}[1]{{\color{green}[WYF: #1]}}
\newcommand{\todo}[1]{{\color{red}[#1]}}

\begin{document}

\title{Supplemental Material: Magic teleportation with generalized lattice surgery}% Force line breaks with \\



\author{Yifei Wang}
\author{Yingfei Gu}
 \email{guyingfei@tsinghua.edu.cn}
\affiliation{Institute for Advanced Study, Tsinghua University, Beijing 100084, China}





\date{\today}% It is always \today, today,
             %  but any date may be explicitly specified



\maketitle



\onecolumngrid

\appendix

\section{A near-term practical implementation}

In this section, we perform a numerical simulation of a near-term practical implementation of our protocol,
namely the preparation of a surface code magic state.
The reason for considering this task instead of performing a $T$ gate on the surface code directly is as follows.
First, the preparation of a magic state is realized by performing a $T$ gate on a $\ket{+}$ state, which is schematically the same as performing a $T$ gate on a generic state.
Second, for near-term demonstrations, state preparation is a simpler task and post selection can be leveraged to reduce the error rate.
Third, we want to emphasize that our protocol can be freely generalized and encorporated into various tasks with other techniques. 

A technical challenge in the simulation is that it requires a large number of qubits and involves non-Clifford gates,
which is beyond the capability of most quantum simulators.
We circumvent this issue by simulating the Clifford gate $T^2 = S$ in the same circuit, and benchmark the difference between $S$ and $T$ gates in small circuits,
which is the stratergy used in \cite{gidneyMagicStateCultivation2024}.
We use STIM \cite{gidney2021stim} for stabilizer simulation and use a Monte Carlo simulation for benchmarking the difference between $S$ and $T$ gates.

The surface code magic state preparation is performed in the following steps:
\begin{enumerate}
    \item Prepare a encoded state $\ket{\overline{+}}$ on the $\llbracket 15,1,3\rrbracket$ quantum Reed-Muller code.
    We first prepare $\ket{+}^{\otimes 15}$ and measure all the $Z$-stabilizers for one round.
    This requires 15 data qubits and 10 ancilla qubits.
    \item At the same time, prepare a distance-3 rotated surface code with 9 data qubits and 8 ancilla qubits.
    \item Measure the combined $Z$-stabilizers in the interface between the two codes for ? rounds.
    The sequence of performing C$Z$ gates for the measurement is designed such that errors can propagate to the QRM code but not the surface code.
    \item Measure all the data qubits of the QRM code in the $X$ basis for ? rounds.
    If the $X$-checks are all satisfied, grow the surface code to a larger (distance-7?) one.
\end{enumerate}



\bibliography{supp}


\end{document}
